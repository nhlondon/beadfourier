 \documentclass[11 pt]{article}
\usepackage[utf8]{inputenc}
\usepackage{graphicx}
\usepackage{color}
\usepackage[english]{babel}
\usepackage{tabu}
\usepackage[letter paper, portrait, margin=1in]{geometry}
\usepackage{enumitem}
\usepackage{subcaption}
\usepackage{csquotes}
\usepackage{float}
\usepackage[font=footnotesize]{caption}
\usepackage{mathtools}
\usepackage{amsmath}
\usepackage{bm}
\usepackage{multicol}
\usepackage{amsfonts}
\usepackage{subcaption}
\usepackage{hyperref}
\usepackage{import}
\hypersetup{
	colorlinks = true,
	citecolor = blue,
	linkcolor = blue,
	linktocpage}
\usepackage[capitalize]{cleveref}
\usepackage{pdfsync}
\newcommand{\crefrangeconjunction}{—}

\graphicspath{{Figures/}}

%\renewcommand{\thesection}{\Roman{section}} 
%\titleformat{\section}
 % {\normalfont\fontsize{12}{15}\bfseries}{\thesection}{1em}{}
%\titlespacing{\section}{0pt}{*0}{*0}
%\titlespacing{\subsection}{0pt}{*0}{*0}
%\titlespacing{\subsubsection}{0pt}{*0}{*0}
\setlength{\belowcaptionskip}{-7pt}


%SetFonts

%SetFonts
\usepackage{pdfsync}
\synctex=1
\definecolor{lightGrey}{rgb}{0.9, 0.9, 0.9}
%\lstset{
%	language={[77]Fortran},
%	keywordstyle=\bfseries\ttfamily\color[RGB]{0,150,0},
%	identifierstyle=\ttfamily,
%	directivestyle=\bfseries\ttfamily\color[RGB]{190,52,211},
%	commentstyle=\ttfamily\color[RGB]{0,0,240},
%	stringstyle=\ttfamily\color[RGB]{240,0,0},
%	showstringspaces=false,
%	basicstyle=\ttfamily\ssmall,
%	numberstyle=\ssmall,
%	numbers=left,
%	stepnumber=1,
%	numbersep=10pt,
%	tabsize=1,
%	breaklines=true,
%	prebreak = \raisebox{0ex}[0ex][0ex]{\ensuremath{\hookleftarrow}},
%	breakatwhitespace=false,
%	aboveskip={.5\baselineskip},
%  columns=fixed,
%  upquote=true,
%  extendedchars=true,
%  escapeinside={(*@}{@*)}
% %frame=single,
% %backgroundcolor=\color{lightGrey},
%}

\newcommand{\Hilight}{\makebox[0pt][l]{\color{lightGrey}\rule[-2pt]{\linewidth}{9pt}}}

\usepackage[compact]{titlesec}
%\renewcommand{\thesection}{\Roman{section}} 
%\titleformat{\section}
 % {\normalfont\fontsize{12}{15}\bfseries}{\thesection}{1em}{}
%\titlespacing{\section}{0pt}{*0}{*0}
%\titlespacing{\subsection}{0pt}{*0}{*0}
%\titlespacing{\subsubsection}{0pt}{*0}{*0}
\setcounter{secnumdepth}{4}

\titleformat{\paragraph}
{\normalfont\normalsize\bfseries}{\theparagraph}{1em}{}
\titlespacing*{\paragraph}
{0pt}{3.25ex plus 1ex minus .2ex}{1.5ex plus .2ex}
\setlength{\belowcaptionskip}{-15pt}
\usepackage[backend=bibtex,
style = phys,
citestyle= phys,
pageranges= false,
biblabel = brackets,
sorting=none]{biblatex}

\newcommand*{\subfilesbibliography}[1]{%
\expandafter\ifx\csname ver@subfiles.cls\endcsname\relax
\expandafter\@secondoftwo
\else
\expandafter\@firstoftwo
\fi
{\bibliography{#1}}
{}%
}

%\bibliography{qcmd}
\graphicspath{{Figures/}}
\pagestyle{myheadings}
\markboth{Nathan London, DL\_POLY f-QCMD}
{Nathan London, DL\_POLY f-QCMD}


\newcommand{\dlpoly}{DL\_POLY Quantum}


\title{Real-time extension to bead-fourier path integrals}
\author{Nathan London}
\date{}							% Activate to display a given date or no date

\begin{document}
\maketitle

\tableofcontents
\newpage

  \begin{equation}
    \label{eq:label}
    Z = \mathrm{Tr} \left[ \mathrm{e}^{-\beta \hat{H}} \right]
  \end{equation}
 
  \begin{equation}
    Z = \int \mathrm{d}x \left\langle x \left| \mathrm{e}^{-\beta \hat{H}} \right| x \right\rangle
  \end{equation}
  
  \begin{equation}	
    Z \propto \int \mathrm{d}\{x_j\} \int \mathrm{d}\{p_j\}\ \mathrm{e}^{-\beta\left[ \sum^{n}_{j=1}\left( \frac{nm}{2\beta^2\hbar^2}
    (x_{j+1}-x_j)^2 + \frac{1}{n}V(x_j)\right)\right]}
  \end{equation}
 
  \begin{equation}
    Z = \oint D(x(u))\mathrm{e}^{-S(x(u))}
  \end{equation}
 
  \begin{equation}
    S(x(u)) = \frac{1}{\hbar}\int_{{0}}^{{\beta\hbar}}\mathrm{d}{u} { \frac{m\dot{x}(u)^2}{2}+ V(x(u))}
  \end{equation}
 
  \begin{equation}
    Z = \int \mathrm{d}x \int_{x(0)=x}^{x(\beta\hbar)=x} D(x(u))\mathrm{e}^{-S(x(u))}
  \end{equation}
  

  \begin{equation}
    x(u) = x + (x' -x) \frac{u}{\beta\hbar} + \sum^{\infty}_{k=1} a_k \sin\left( \frac{k\pi u}{\beta\hbar}\right)
  \end{equation}
 
  $u_j = (j-1) \frac{\beta\hbar}{n}$
  $x(u_j) = x_j$ with $x(u_{n+1}) = x_1$
  \begin{equation}
    Z = \int \mathrm{d}\{x_j\}\int_{x_1}^{x_2} D(x(u))\dots\int_{x_n}^{x_1} D(x(u))
    \mathrm{e}^{ -\sum_{j=1}^n S_j(x(u))} 
  \end{equation}
 
  \begin{equation}
    S_j(x(u)) = \frac{1}{\hbar}\int_{u_j}^{u_{j+1}}\mathrm{d}{u}  \frac{m\dot{x}(u)^2}{2}+ V(x(u))
  \end{equation}
 
  \begin{equation}
    x_j(u) = x_j + \frac{(x_{j+1}-x_j)(u-u_j)}{u_{j+1}-u_j} + \sum^{\infty}_{k=1} a_{jk}\sin\left(
      \frac{k\pi(u-u_j}{u_{j+1}-u-j}\right)
  \end{equation}
  

  \begin{equation}
    \dot{x}(u) = \frac{dx(u)}{du} = \frac{x_{j+1}-x_j}{u_{j+1}-u_j} + \sum_k a_{jk} \frac{k\pi}{u_{j+1}-u_j}
    \cos\left( \frac{k\pi(u-u_j)}{u_{j+1}-u_j}\right)
  \end{equation}
 
  $\xi = \frac{u-u_j}{u_{j+1}-u_j}$, $u_{j+1}-u_j = \frac{\beta\hbar}{n}$
  \begin{multline}
    \dot{x}(u)^2 = \left(\frac{(x_{j+1}-x_j)n}{\beta\hbar}\right)^2 + 
    2\frac{(x_{j+1}-x_j)n}{\beta\hbar} 
    \sum_k a_{jk} \frac{k\pi n}{\beta\hbar}
    \cos\left( k\pi\xi\right) \\
    +\left(\sum_k a_{jk} \frac{k\pi n}{\beta\hbar}
    \cos\left(k\pi\xi\right)\right)^2
  \end{multline}
  

  $\int_0^1 \mathrm{d}\xi \cos(k\pi\xi) = 0$
  \begin{equation}
    \begin{split}
      \frac{1}{\hbar}\int_{u_j}^{u_{j+1}} \mathrm{d}u \frac{m\dot{x}(u)^2}{2} 
        &= \frac{m\beta\hbar}{2\hbar n}
        \int_0^1\mathrm{d}\xi\ \dot{x}(\xi)^2 \\
        &= \frac{mn}{2\beta\hbar^2}\left[(x_{j+1}-x_j)^2 + \sum_k \frac{(k\pi)^2}{2}a_{jk}^2\right]
    \end{split}
  \end{equation}
   
    \begin{equation}
      \frac{1}{\hbar}\int_{u_j}^{u_{j+1}}\mathrm{d}u\ V(x(u)) = \frac{\beta}{n}\int_0^1\mathrm{d}\xi\ V(x_j(\xi))
    \end{equation}
    
  \begin{equation}
    Z = C(\beta)\int\mathrm{d}\{x_j\}\int\mathrm{d}\{a_{jk}\}\ \mathrm{e}^{-\beta H(x_j,a_{jk})}
  \end{equation}

  
  \begin{multline}
    H(x_j,a_{jk})= \sum_{j=1}^{n} \left[ \frac{mn}{2\beta^2\hbar^2}\left((x_{j+1}-x_j)^2 + \sum^{k_\mathrm{max}}_{k=1}
      \frac{(k\pi)^2}{2}a_{jk}^2\right) \right. \\ \left.   + \frac{1}{n}\int_0^1\mathrm{d}\xi\ V(x_j(\xi))\right]
  \end{multline}

  \begin{equation}
    C(\beta) = \left( \frac{mn}{2\beta\hbar^2}\right)^{ \frac{n}{2}(1+k_{\mathrm{max}})}
    \frac{k_\mathrm{max}!}{\sqrt{2}}
  \end{equation}
 
  \begin{equation}
    H \to H + \sum^{n}_{j=1} \left[ \frac{p_j^2}{2m} + \sum^{k_\mathrm{max}}_{k=1} \frac{p_{jk}^2}{2m_{k}}\right]
  \end{equation}

  The equations of motion for the bead-fourier system can be found as,
  \begin{equation}
    \frac{ \partial p_{j}}{ \partial t} = -\frac{ \partial H}{ \partial x_{j}},
    \label{eq:eom-p-bead}
  \end{equation}
  \begin{equation}
    \frac{ \partial x_{j}}{ \partial t} = \frac{ \partial H}{ \partial p_{j}} = \frac{p_{j}}{m},
    \label{eq:eom-x-bead}
  \end{equation}
  \begin{equation}
    \frac{ \partial p_{jk}}{ \partial t} = -\frac{ \partial H}{ \partial a_{jk}},
    \label{eq:eom-p-fourier}
  \end{equation}
  and
  \begin{equation}
    \frac{ \partial a_{jk}}{ \partial t} = \frac{ \partial H}{ \partial p_{jk}} = \frac{p_{jk}}{m_{k}},
    \label{eq:eom-a-fourier}
  \end{equation}
  with the derivative of the Hamiltonian with respect to the bead positions being
  \begin{align} 
  %\begin{equation}
    \frac{ \partial H}{ \partial x_j} 
    &= \frac{ \partial }{ \partial x_j} \left[ \sum^{n}_{j=1} \frac{1}{2}\omega_n^2
    (x_{j+1}-x_j)^2 \right] + \frac{ \partial }{ \partial x_j} \left[ \sum^{n}_{j=1} \frac{1}{n} \int_0^1 
    \mathrm{d}\xi\ V[x_j(\xi)] \right] \\
    &= \omega_n^2(2x_j-x_{j+1}-x_{j-1})
    + \frac{1}{n}\left[
    \int_0^1 \mathrm{d}\xi\ \frac{ \partial V[x_j(\xi)]}{ \partial x_j(\xi)} \frac{ \partial x_j(\xi)}{ \partial x_j}
    +\int_0^1 \mathrm{d}\xi\ \frac{ \partial V[x_{j-1}(\xi)]}{ \partial x_{j-1}(\xi)} \frac{ \partial x_{j-1}(\xi)}{
    \partial x_j}\right]\\
    &= \omega_n^2(2x_j-x_{j+1}-x_{j-1})
    + \frac{1}{n}\left[
    \int_0^1 \mathrm{d}\xi\ \frac{ \partial V[x_j(\xi)]}{ \partial x_j(\xi)}(1-\xi)
    +\int_0^1 \mathrm{d}\xi\ \frac{ \partial V[x_{j-1}(\xi)]}{ \partial x_{j-1}(\xi)}\xi
    \right].
  %\end{equation}
    \label{eq:part-x}
  \end{align}

  The derivative of the Hamiltonian with respect to the Fourier amplitudes is
\begin{align}
  \frac{ \partial H}{ \partial a_{jk}}  
    &= \frac{1}{2} \omega_{n}^{2}(k\pi)^2 a_{jk} + \frac{1}{n}\int_{0}^{1} \mathrm{d} \xi\
  \frac{ \partial V[x_{j}(\xi)]}{ \partial x_{j}(\xi)} \frac{ \partial x_{j}(\xi)}{ \partial a_{jk}}\\
    &= \frac{1}{2} \omega_{n}^{2}(k\pi)^2 a_{jk}+ \frac{1}{n}\int_{0}^{1} \mathrm{d} \xi\
  \frac{ \partial V[x_{j}(\xi)]}{ \partial x_{j}(\xi)} \sin(k\pi\xi).
  \label{eq:part-a}
\end{align}

\begin{equation}
  \sum^{n}_{j=1} \frac{ \partial H}{ \partial x_{j}} = \frac{1}{n} \sum^{n}_{j=1} \int_{0}^{1} \mathrm{d} \xi\
  \frac{ \partial V[x_{j}(\xi)]}{ \partial x_{j}}
  \label{eq:sum-part-x}
\end{equation}

  \end{document} 

